\onehalfspacing
\justify
\setlength{\parindent}{2em}
\chapter{Introduction}\label{chap:intro}

%%%%%%%%%%%%%%%%%%%%%%%%%%%%%%%%%%%%%%%%
\section{Models and Motivation}\label{sec:models}
%%%%%%%%%%%%%%%%%%%%%%%%%%%%%%%%%%%%%%%%

There is strong indirect evidence for the existence of some form of dark matter in the Universe including anomalies in galactic rotation curves, weak lensing of distant galaxies, observations of the bullet cluster, and Cosmic Microwave Background parameter measurements \cite{Bertone:2004pz, Cirelli:2012tf, Gaskins:2016cha, Slatyer:2017sev}. Nonetheless, the nature of dark matter is still unknown and thermal relics associated with the electroweak scale are under increasing pressure from the LHC as well as from direct and indirect detection experiments~\cite{Aaboud:2017phn,Sirunyan:2018xlo,Aprile:2018dbl,Akerib:2016vxi,Cui:2017nnn,TheFermi-LAT:2017vmf,Aghanim:2018eyx,Escudero:2016gzx,Arcadi:2017kky,Roszkowski:2017nbc,Athron:2017kgt,Arcadi:2019lka,Blanco:2019hah}. Motivated by the fact that direct detection experiments are considerably less sensitive to sub-GeV dark matter particles, attention has naturally turned to lower mass alternatives~\cite{Essig:2013lka,Alexander:2016aln,Battaglieri:2017aum,Beacham:2019nyx}.

From a theoretical perspective, MeV-scale thermal dark matter candidates were shown to be viable some years ago \cite{Boehm:2003hm,Boehm:2003bt,Feng:2008ya} and, since then, a large number of MeV-scale dark matter models have appeared in the literature, see e.g.~\cite{Boehm:2006mi,Farzan:2009ji,Farzan:2011ck,Batell:2017cmf,Ballett:2019cqp,Lamprea:2019qet,Blennow:2019fhy,Krnjaic:2015mbs,Bondarenko:2019vrb,Hochberg:2014dra,Agrawal:2014ufa,Kamada:2018zxi,Knapen:2017xzo,Hall:2009bx,Chu:2011be,Hambye:2019dwd,Dvorkin:2019zdi,Evans:2019vxr}. Experimentally, with the aim of testing as many scenarios as possible \cite{Bertone:2018xtm}, a complementary program has been developed to test the possible existence of MeV-scale dark matter particles and potential companions Beyond the Standard Model (BSM)~\cite{Essig:2013lka,Alexander:2016aln,Battaglieri:2017aum,Beacham:2019nyx}. Light dark matter particles and their potential mediators with the dark sector have been searched for at particle colliders~\cite{Borodatchenkova:2005ct,Batell:2009yf,Fox:2011fx,Essig:2013vha,Lees:2014xha,Babusci:2015zda,Lees:2017lec,Aaij:2017rft}, beam dump experiments~\cite{Bjorken:2009mm,Batell:2009di,Andreas:2012mt}, neutrino experiments~\cite{PalomaresRuiz:2007eu,Harnik:2012ni,deNiverville:2012ij,Batell:2014yra,Klop:2018ltd,Kelly:2019wow}, neutrino telescopes~\cite{Kamada:2015era,Arguelles:2017atb,Alvey:2019jzx}, as well as in direct~\cite{Essig:2015cda,Lee:2015qva,Derenzo:2016fse,Essig:2017kqs,Agnese:2018col,Agnes:2018oej,Abramoff:2019dfb,Aprile:2019xxb} and indirect~\cite{Slatyer:2015jla,Essig:2013goa,Bartels:2017dpb} dark matter detection experiments. Searches for light BSM species are not only carried out in terrestrial experiments, but a variety of astrophysical~\cite{Raffelt:1996wa,Dreiner:2013mua,Chang:2018rso,DeRocco:2019jti,Farzan:2002wx,Heurtier:2016otg} and cosmological~\cite{Xu:2018efh,Campo:2017nwh,Berlin:2018sjs,Wilkinson:2014ksa,Bertoni:2014mva,Vogel:2013raa,Escudero:2019gzq,Dolgov:2013una} constraints have been also derived on states with masses at the MeV scale. Up to now, however, all these searches have been unsuccessful. Future, ongoing and planned experiments are expected to cut into relevant regions of parameter space and perhaps yield a signal~\cite{Essig:2013lka,Alexander:2016aln,Battaglieri:2017aum,Beacham:2019nyx,Kou:2018nap,Ariga:2019ufm,Alekhin:2015byh,Chou:2016lxi,Akesson:2018vlm}.

%%%%%%%%%%%%%%%%%%%%%%%%%%%%%%%%%%%%%%%%
\section{Cosmology and Nucleosynthesis}
%%%%%%%%%%%%%%%%%%%%%%%%%%%%%%%%%%%%%%%%

Big Bang Nucleosynthesis (BBN) has been widely used as a probe of new physics~\cite{Sarkar:1995dd,Iocco:2008va,Pospelov:2010hj}. BBN occurred when the Universe was about three minutes old, in the temperature range $10\,\text{keV}  \lesssim T \lesssim 1\,\text{MeV}$, and therefore represents a key stage of the Universe that new states at the MeV scale can affect. Given the excellent agreement between observations and the Standard Model (SM) prediction of the primordial light nuclei abundances~\cite{pdg}, strong constraints can be set on the masses and properties of new light particles. Similarly, the agreement of Cosmic Microwave Background (CMB) observations with a vanilla $\Lambda$CDM Universe~\cite{Aghanim:2018eyx} can be used to set strong constraints on light new physics.

In Chap. \ref{chap:bbn}, we perform an exhaustive and robust analysis of the cosmological implications of MeV-scale particles that are thermally coupled to electrons, neutrinos or both in the early Universe. This has been studied in the past by a number of groups~\cite{Kolb:1986nf,Serpico:2004nm,Boehm:2013jpa,Nollett:2013pwa,Nollett:2014lwa,Boehm:2012gr,Ho:2012ug,Wilkinson:2016gsy,Depta:2019lbe,Escudero:2018mvt}, but here we update and upgrade the constraints by:
\begin{itemize}[leftmargin=0.5cm,itemsep=0pt]
\item Using up-to-date measurements of the primordial element abundances \cite{pdg} and Planck 2018 CMB observations \cite{Aghanim:2018eyx}.
\item Accurately accounting for the early Universe evolution in the presence of MeV-scale states following \cite{Escudero:2018mvt,Escudero:2019new}.
\item Using the state-of-the-art Big Bang Nucleosynthesis code \texttt{PRIMAT} \cite{Pitrou:2018cgg}, which outputs the most accurate theoretical predictions for the helium and deuterium abundances to date. \texttt{PRIMAT} accounts for a variety of effects, such as up-to-date nuclear reaction rates, finite temperature corrections, incomplete neutrino decoupling and several other effects relevant to the proton-to-neutron conversion rates.
\item Performing a pure BBN analysis on light MeV-scale states. Namely, we set a bound on the masses of different species by using only the primordial helium and deuterium abundances and by marginalizing over any possible value of the baryon energy density.
\end{itemize}

%%%%%%%%%%%%%%%%%%%%%%%%%%%%%%%%%%%%%%%%
\section{Direct Detection of sub-GeV Dark Matter}
%%%%%%%%%%%%%%%%%%%%%%%%%%%%%%%%%%%%%%%%

As mentioned above, the experimental search for a direct detection signature~\cite{Goodman:1984dc} is still ongoing. Light sub-GeV dark matter, in particular, has become a prime target of such activities and encompasses a wider variety of possibilities for new physics beyond the Standard Model~\cite{Battaglieri:2017aum}.

Acquiring sensitivity to sub-GeV dark matter typically requires detectors able to pick up lower recoil energies; a reduced momentum is expected for lighter masses given that the galactic velocity of dark matter is $\mathcal{O}(10^{-3}) c$. However, Refs.~\cite{Bringmann:2018cvk, Ema:2018bih} recently showed that light dark matter interacting with nucleons or electrons necessarily leads to an energetic flux due to cosmic rays colliding elastically with dark matter in the interstellar medium. This up-scattered dark matter flux may then have enough energy to be detectable in direct detection experiments such as XENON1T (previously thought to be sensitive only to $\mathcal{O}(10-100)$ GeV dark matter)~\cite{Aprile:2018dbl}, as well as other dark matter or neutrino detectors~\cite{Bringmann:2018cvk, Ema:2018bih}~\footnote{See also Refs.~\cite{Kouvaris:2016afs, Ibe:2017yqa, Dolan:2017xbu} for ways of extending the direct detection sensitivity to lighter dark matter masses, and Refs.~\cite{Kouvaris:2015nsa,An:2017ojc,Emken:2017hnp} for solar sources of energetic dark matter flux.}.

In Chap. \ref{chap:cr}, we point out another generic (albeit not irreducible) source of light dark matter flux. If mesons decay partially into dark matter, as could happen through the same coupling to nucleons that enables direct detection, then the mesons generated in inelastic cosmic ray collisions will also produce an energetic flux of dark matter. This may be viewed as a continuous cosmic beam dump experiment. It naturally provides a preexisting light dark matter source for experiments that would otherwise be insensitive to them. The different detector targets, exposure, and source geometry involved then enable distinctive opportunities relative to dedicated beam dump experiments. Indeed, we shall see that XENON1T~\cite{Aprile:2018dbl} and the future LZ experiment~\cite{Akerib:2018lyp} set competitive limits for light mediators in comparison to MiniBooNE~\cite{Aguilar-Arevalo:2018wea}. Moreover, unlike the upscattering mechanism that relies on a relic dark matter density, inelastic cosmic ray collisions can also produce other long-lived hidden sector particles, thus extending the possibilities for direct detection coverage of light sectors beyond dark matter.

In particular, we provide a first estimate of the dark matter flux from the aforementioned cosmic ray mechanism, taking into account their attenuation through the Earth. As an example of its application, we then place current and projected limits from XENON1T and LZ. We do this generally for a model-independent parametrisation of spin-independent cross-section vs dark matter mass and vs the meson branching ratio into dark matter. Finally, we consider a specific model in which the dark sector mediator is a hadrophilic scalar particle~\cite{Batell:2018fqo}.

%%%%%%%%%%%%%%%%%%%%%%%%%%%%%%%%%%%%%%%%
\section{Neutrinos at IceCube}
%%%%%%%%%%%%%%%%%%%%%%%%%%%%%%%%%%%%%%%%

There are a number of issues with the vanilla Standard Model of particle physics \cite{Bonnet2012, Farzan2011, Bednyakov2007, Kubo2006, Davidson2002, Ma2001, Yao2018}. One particularly prominent qustion is the origin of neutrino masses. In the pure $\rm{SU}(3) \times \rm{SU}(2) \times \rm{U}(1)$ SM, neutrinos are massless. This is in contradiction to evidence for neutrino oscillations \cite{Fukuda:1998mi}. Such a hint of new physics beyond the SM is of interest to model builders looking for extensions to the current known particle content \cite{Ma1998,Ma2001,Ma2006a}.

In Chap. \ref{chap:neutrinos}, we consider an effective model of particle Dark Matter that consistently generates neutrino masses \cite{Boehm,Farzan2009, Ma2006, Franarin2018, Farzan2010, Artamonov2016, Farzan2010a, Farzan2011, Ambrosino2009, Farzan2014, Ma2006a, Boehm2006, Serpico2004, Boehm2004, Boehm2003, Ma1998}. In particular, we look to place complementary bounds to those discussed in Section \ref{sec:models} using the IceCube neutrino experiment. IceCube is a relatively new facility located at the South Pole, which detects high energy astrophysical neutrinos \cite{Kelly, Padovani2018, IceCube, IceCube2018, Ackermann2018, Difranzo2015, Hooper2018, Ioka2014}. On the $22^{\textrm{nd}}$ September $2017$, IceCube detected a $290\,\textrm{TeV}$ muon neutrino. After analysis of the trajectory of the event, it is generally believed that this neutrino came from the blazar TXS $0506$+$056$. This active galactic nuclei is located approximately $1.3 \,\textrm{Gpc}$ away in comoving distance co-ordinates. The fact that the blazar neutrino propagates over gigaparsecs of cosmological distance means that we may be able to compute the mean free path of the neutrino within the new framework and find something comparable with this distance. Note that this is non-trivial in the sense that the SM prediction for the neutrino mean free path after the neutrino sector decouples from the photon bath is $\mathcal{O}(10^{11}) \, \textrm{Gpc}$.

The result of this study are competitive exclusion limits on the complex scalar Dark Matter candidate in this model. The constraint is complimentary (and slightly weaker) to current constraints from Big Bang Nucleosynthesis and the Cosmic Microwave Background such as those detailed in Chap. \ref{chap:bbn}. We also explore how future higher energy events could improve this bound.

%%%%%%%%%%%%%%%%%%%%%%%%%%%%%%%%%%%%%%%%
\section{Summary of the Work}
%%%%%%%%%%%%%%%%%%%%%%%%%%%%%%%%%%%%%%%%

To summarise, the rest of the work is organised as follows. In Chap. \ref{chap:bbn}, we present the bounds coming from Big Bang Nucleosynthesis and the Cosmic Microwave Background. We also consider the sensitivity of future CMB experiments such as the Simons Observatory and CMB-S4. In Chap. \ref{chap:cr}, we calculate the expected flux of high-momenta dark matter that may arise in inelastic collisions of cosmic rays with the atmosphere. In turn, limits are placed on the spin-indepdent Dark Matter-Nucleon cross section using the XENON1T and LZ experiments. Finally, in Chap. \ref{chap:neutrinos}, the propagation of neutrinos across a few gigaparsecs of cosmological distance is considered within the framework of an effective theory that radiatively generates neutrino masses. Observations at the IceCube neutrino telescope are then used to constrain the model and forecasts are made regarding the impact of higher energy neutrino measurements.