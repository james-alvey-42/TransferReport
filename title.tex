\begin{titlepage}

\noindent
\titlefont \vspace*{3cm} \\ Cosmological Constraints on MeV-scale Dark Sectors \par

\vspace{0.3cm}
\epigraph{\justify{\textbf{Abstract} \\ In this work we present new and refined cosmological constraints on MeV-scale dark sectors and various well-motivated particle physics models. New light states thermally coupled to the Standard Model plasma can alter the expansion history of the Universe and impact the synthesis of the primordial elements. We present precise and robust constraints from Big Bang Nucleosynthesis (BBN) and the Cosmic Microwave Background (CMB). We find that BBN observations alone set a lower bound on the thermal dark matter mass to be $m_\chi > 0.4 \, \mathrm{MeV}$ at $2\sigma$. The reach of future CMB experiments is also considered. 

On a more terrestrial note, we reinterpret the results of direct detection experiments which rely on nuclear recoils. Such experiments lose sensitivity for sub-GeV dark matter although this is recovered if there is an additional, sub-dominant source of particles with higher momenta. We investigate the possibility that decays of mesons from inelastic cosmic ray collisions in the atmosphere could act as such a source. The resulting constraints are then presented and mapped onto a concrete particle physics model. 

Finally, we look at a well-motivated model that links neutrino masses and scalar dark matter. Bounds on parameters within the model are placed using measurements from the IceCube neutrino telescope and future sensitivity is forecast.}}%
\null\vfill
\vspace*{0.1cm}
\noindent
\hfill
\begin{minipage}{0.5\linewidth}
    \begin{flushright}
        \printauthor
    \end{flushright}
\end{minipage}
%
\begin{minipage}{0.02\linewidth}
    \begin{flushright}
    \rule{0.1pt}{140pt}
    \end{flushright}
\end{minipage}
\titlepagedecoration
\end{titlepage}