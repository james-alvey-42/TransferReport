\usepackage[margin=1.2in]{geometry}

\usepackage{amsmath}
\usepackage{amssymb}
\usepackage{amsthm}
\usepackage{mathtools}
\usepackage[protrusion=true]{microtype}
\usepackage{breqn}
\usepackage{slashed}
\usepackage{breqn}
\usepackage{setspace}
\usepackage{framed}
\usepackage{tikz}
\usepackage{epigraph}
\usepackage[document]{ragged2e}
\usepackage[T1]{fontenc}
\usepackage{titlesec}
\usepackage{cases}
\usepackage{multirow}
\usepackage[nottoc,notlot,notlof]{tocbibind}
\usepackage{tocloft}
\addtolength{\cftchapnumwidth}{10pt}
\addtolength{\cftsecindent}{10pt}
\addtolength{\cftsecnumwidth}{0pt}
\addtolength{\cftsubsecindent}{10pt}
\addtolength{\cftsubsecnumwidth}{0pt}
\renewcommand\cftchapafterpnum{\vskip3.5pt}
\renewcommand\cftsecafterpnum{\vskip3.5pt}
\renewcommand\cftsubsecafterpnum{\vskip3.5pt}

\newcommand{\JA}[1]{{\bf \color{cyan}{[JA: #1]}}}
\newcommand{\Abs}[1]{\left|#1\right|}
\newcommand{\mO}{\mathcal{O}}
\newcommand{\mM}{\mathcal{M}}
\newcommand{\dagg}{^{\dagger}}
\newcommand{\ud}{\mathrm{d}}
\newcommand{\upd}[1]{\mathrm{d}#1\,}
\renewcommand\vec{\mathbf}
\newcommand{\normord}[1]{\raisebox{0.5pt}{:}\,#1\,\raisebox{0.5pt}{:}}
\newcommand{\pr}{^{\prime}}
\newcommand{\nhat}{\hat{\bm{n}}}
\newcommand{\hamilt}{\mathcal{H}}
\newcommand{\mA}{\mathcal{A}}
\newcommand{\mW}{\mathcal{W}}
\newcommand{\mN}{\mathcal{N}}
\newcommand{\mD}{\mathcal{D}}
\newcommand{\mS}{\mathcal{S}}
\newcommand{\mL}{\mathcal{L}}
\newcommand{\mC}{\mathcal{C}}
\newcommand{\mT}{\mathcal{T}}
\newcommand{\mZ}{\mathcal{Z}}
\newcommand{\mR}{\mathcal{R}}
\newcommand{\II}{\mathbb{I}}
\newcommand{\RR}{\mathbb{R}}
\newcommand{\ZZ}{\mathbb{Z}}
\newcommand{\CC}{\mathbb{C}}
\newcommand{\FF}{\mathbb{F}}
\newcommand{\lie}[1]{\mathcal{L}\left(#1\right)}
\newcommand{\set}[1]{\left\{#1\right\}}
\newcommand{\SO}[1]{\textrm{SO}\left(#1\right)}
\newcommand{\SU}[1]{\textrm{SU}\left(#1\right)}
\newcommand{\Orth}[1]{\textrm{O}\left(#1\right)}
\newcommand{\Uni}[1]{\textrm{U}\left(#1\right)}
\newcommand{\paraskip}{\vspace{10pt}}
\newcommand{\del}{\partial}
\newcommand{\TeG}{\mathcal{T}_e(\mathscr{G})}
\newcommand{\TpM}{\mathcal{T}_p(\mathcal{M})}
\newcommand{\TpMs}{\mathcal{T}^{\star}_p(\mathcal{M})}
\newcommand{\etamn}[1]{\eta#1{\mu \nu}}
\newcommand{\group}{\mathscr{G}}
\newcommand{\alge}{\mathfrak{g}}
\newcommand{\twobytwo}[4]{\begin{pmatrix}#1&#2 \\ #3&#4 \end{pmatrix}}
\newcommand{\thrbythr}[3]{\begin{pmatrix}#1 \\ #2 \\ #3\end{pmatrix}}
\newcount\colveccount
\newcommand*\colvec[1]{
        \global\colveccount#1
        \begin{pmatrix}
        \colvecnext
}
\def\colvecnext#1{
        #1
        \global\advance\colveccount-1
        \ifnum\colveccount>0
                \\
                \expandafter\colvecnext
        \else
                \end{pmatrix}
        \fi
}
\newcommand{\tr}{\text{Tr}}

\usepackage[inline]{enumitem}
\renewcommand\labelitemi{\raisebox{0.25ex}{\tiny$\bullet$}}

\usepackage{xcolor}
\usepackage[colorlinks]{hyperref}
\hypersetup{colorlinks=true, urlcolor=magenta, citecolor=magenta, runcolor=black, menucolor=black, filecolor=black, anchorcolor=black, linkcolor=magenta}

\usepackage{booktabs}

\usepackage{caption}
\captionsetup[figure]{labelsep=quad, labelfont=bf, width=\linewidth}

\usepackage{graphicx}
\graphicspath{{Figures/}}

\usepackage[sort&compress, numbers]{natbib}
\setlength{\bibsep}{1pt}
\bibliographystyle{JHEP}
\renewcommand{\bibname}{References}
\newcommand{\eprint}[1]{\href{http://arxiv.org/abs/#1}{#1}}

\numberwithin{equation}{chapter}
\numberwithin{figure}{chapter}
\numberwithin{table}{chapter}

\renewcommand\thechapter{\Roman{chapter}}
\renewcommand\thesection{\arabic{chapter}.\arabic{section}}
\renewcommand\thesubsection{\arabic{chapter}.\arabic{section}.\arabic{subsection}}
\renewcommand\thefigure{\arabic{chapter}.\arabic{figure}}
\renewcommand\theequation{\arabic{chapter}.\arabic{equation}}

\renewcommand\epigraphflush{flushright}
\renewcommand\epigraphsize{\normalsize}
\setlength\epigraphwidth{0.7\textwidth}

\definecolor{titlepagecolor}{cmyk}{0, 0, 0, 1}

\DeclareFixedFont{\titlefont}{T1}{ppl}{b}{it}{0.25in}

\makeatletter                       
\def\printauthor{%                  
    {\large \@author}}              
\makeatother

% The following code is borrowed from: https://tex.stackexchange.com/a/86310/10898

\newcommand\titlepagedecoration{%
\begin{tikzpicture}[remember picture,overlay,shorten >= -10pt]

\coordinate (aux1) at ([yshift=-15pt]current page.north east);
\coordinate (aux2) at ([yshift=-410pt]current page.north east);
\coordinate (aux3) at ([xshift=-4.5cm]current page.north east);
\coordinate (aux4) at ([yshift=-150pt]current page.north east);

\begin{scope}[titlepagecolor!40,line width=12pt,rounded corners=12pt]
\draw
  (aux1) -- coordinate (a)
  ++(225:5) --
  ++(-45:5.1) coordinate (b);
\draw[shorten <= -10pt]
  (aux3) --
  (a) --
  (aux1);
\draw[opacity=0.6,titlepagecolor,shorten <= -10pt]
  (b) --
  ++(225:2.2) --
  ++(-45:2.2);
\end{scope}
\draw[titlepagecolor,line width=8pt,rounded corners=8pt,shorten <= -10pt]
  (aux4) --
  ++(225:0.8) --
  ++(-45:0.8);
\begin{scope}[titlepagecolor!70,line width=6pt,rounded corners=8pt]
\draw[shorten <= -10pt]
  (aux2) --
  ++(225:3) coordinate[pos=0.45] (c) --
  ++(-45:3.1);
\draw
  (aux2) --
  (c) --
  ++(135:2.5) --
  ++(45:2.5) --
  ++(-45:2.5) coordinate[pos=0.3] (d);   
\draw 
  (d) -- +(45:1);
\end{scope}
\end{tikzpicture}%
}

\definecolor{gray75}{gray}{0.75}
\newcommand{\hsp}{\hspace{20pt}}
\titleformat{\chapter}[hang]{\Huge\bfseries}{\thechapter\hsp\textcolor{gray75}{|}\hsp}{0pt}{\Huge\bfseries}

